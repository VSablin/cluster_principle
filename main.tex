\documentclass[aps,pra,reprint,amsmath,amssymb]{revtex4-1}

\usepackage{subfigure,dcolumn}
\usepackage[T2A,T1]{fontenc}
\usepackage[english]{babel}

\usepackage{braket}
\usepackage{graphicx}
\usepackage[colorinlistoftodos]{todonotes}
\usepackage[utf8]{inputenc}

% The following package will be used to typeset the LaTeX codes and is not a necessity to this template
\usepackage{listings}
\lstloadlanguages{[LaTeX]TeX}
\lstset{language=[LaTeX]TeX,keywordstyle=\color{red},showspaces=true,breaklines=true,breakatwhitespace=true,basicstyle=\small\tt,commentstyle=\color{white},frame=single,framerule=0pt,backgroundcolor=\color{yellow}}


\begin{document}


\title{Causality and the N-photon Scattering matrix in waveguide QED}

\author{...}
\affiliation{...}
\email[Corresponding author, ]{The name, complete address, telephone number, and e-mail address of the author to whom correspondence and proofs should be sent.}


\begin{abstract}
The scattering matrix is one of the most fundamental objects for
describing particle processes. It connects the far past state with the
far future state, where the particles are well described by a free
Hamiltonian but they interact in some nontrivial way
for mid times. It can always be split into two parts: the linear part,
where each particle is scattered independently, and the nonlinear one,
which gives information about the interaction among the
particles. Here, we study the linear part of the scattering matrix for
$N$ photons impinging on a local system with several stable states. 
bla, bla
\end{abstract}


\keywords{Quantum optics, scattering matrix, and few-photon photonics.}

\maketitle


\section{Introduction}

In the last years, a lot of effort has been devoted to the study of interacting quantum systems. The accurate control we have in the laboratory allows to perform quantum tasks such as quantum computing protocols, quantum communications, quantum simulations, etc. One of the most promising fields to deal with this kind of physics is waveguide QED (wQED), where few-photon states propagate through a one-dimensional system and interact with some few-level quantum systems. \cite{weinberg1995,fan10,Xu2015,Xu2016,Sanchez-Burillo2015,Sanchez-Burillo2016}


\section{Model in quantum photonics} 
\begin{figure}
\includegraphics[scale=0.25]{input.pdf}
\caption{Two-photon input state impinging on a $\lambda$ atom. The state $e$ is an unstable state which decays to the stable ones, $g_1$ and $g_2$.}
\end{figure}

\appendix

\begin{figure}
\includegraphics[scale=0.25]{lower_contour.pdf}
\caption{Integration contour for Eq. (...). The red points are the poles of the
integrand. The values of the real parts are arbitrary.}
\end{figure}

\begin{figure}
\includegraphics[scale=0.25]{upper_contour.pdf}
\caption{Integration contour for Eq. (...). The red points are the poles of the
integrand. The values of the real parts are arbitrary.}
\end{figure}


\section{Two-photon scattering matrix in position space}

Here, we compute the linear part of the two-photon scattering matrix in position space. We will see it is different from $S^0$ for systems in which the stable state is unique. The only difference will be a $\theta$ function ensuring the order in which the photons leave the scatterer, so that the system fulfils causality. However, $\theta$ will disappear if there is just a stable state. 

\bibliographystyle{apsrev4-1}
\bibliography{bib_cluster}



\end{document}