\documentclass[aps,pra,reprint,amsmath,amssymb]{revtex4-1}

\usepackage{subfigure,dcolumn}
\usepackage[T2A,T1]{fontenc}
\usepackage[english]{babel}

\usepackage{braket}
\usepackage{graphicx}
\usepackage[colorinlistoftodos]{todonotes}
\usepackage[utf8]{inputenc}

% The following package will be used to typeset the LaTeX codes and is not a necessity to this template
\usepackage{listings}
\lstloadlanguages{[LaTeX]TeX}
\lstset{language=[LaTeX]TeX,keywordstyle=\color{red},showspaces=true,breaklines=true,breakatwhitespace=true,basicstyle=\small\tt,commentstyle=\color{white},frame=single,framerule=0pt,backgroundcolor=\color{yellow}}


\begin{document}


\title{Cluster principle + N-photon scattering matrix}

\author{...}
\affiliation{...}
\email[Corresponding author, ]{The name, complete address, telephone number, and e-mail address of the author to whom correspondence and proofs should be sent.}


\begin{abstract}
The structure of the linear part of the N-photon scattering matrix should be blablabla
\end{abstract}


\keywords{Quantum optics, scattering matrix, and few-photon photonics.}

\maketitle


\section{Introduction}

\cite{fan10,Xu2015,Xu2016,Sanchez-Burillo2016}

\bibliographystyle{apsrev4-1}
\bibliography{bib_cluster}



\end{document}